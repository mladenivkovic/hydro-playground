\newpage
%======================================================
\section{Getting Started}
%======================================================



%======================================================
\subsection{Getting the Code}
%======================================================


The code is hosted on a github repository:
\url{https://github.com/mladenivkovic/hydro-playground}. You can get it by
cloning the repository using either

\begin{lstlisting}
$ git clone https://github.com/mladenivkovic/hydro-playground.git
\end{lstlisting}


using the \jargon{https} protocol, or

\begin{lstlisting}
$ git clone git@github.com:mladenivkovic/hydro-playground.git
\end{lstlisting}

using \jargon{ssh}.








%======================================================
\subsection{Getting and Installing the Python Module}
%======================================================


The entire \mhutils python module is stored within this repository as a git
submodule of its own
repository\footnote{\url{https://github.com/mladenivkovic/mesh_hydro_utils}}.

Once you've cloned the mesh-hydro repository, you'll also need to tell git to
grab the submodules using

\begin{lstlisting}
$ git submodule init
$ git submodule update
\end{lstlisting}

When completed successfully, the directory \verb|./python_module| should now
contain some files. We now need to install this python module.

The easiest way is to navigate into the directory and install it locally using
e.g. \code{pip}:

\begin{lstlisting}
$ cd python_module
$ pip install -e .
\end{lstlisting}

On your local machine, that should work just fine. If you are trying to install
this python package some place where your permissions are restricted, e.g. on a
cluster or HPC facility, then \code{pip} may be prohibited from installing this
way. In those cases, you can try

\begin{lstlisting}
$ pip install --user -e .
\end{lstlisting}

first. If that doesn't work, a reliable failsafe is to create a python virtual
environment and install it there. (In fact, this should be the preferred
method.) You can do this by:

\begin{lstlisting}
$ python3 -m venv $HOME/virtualenv/some_name
$ source $HOME/virtualenv/some_name/bin/activate
$ cd ./python_module
$ pip install -e .
\end{lstlisting}

You may choose the path \verb|$HOME/virtualenv/some_name| freely to your liking.
Note that every time you start a new session, you'll need to run the command

\begin{lstlisting}
$ source $HOME/virtualenv/some_name/bin/activate
\end{lstlisting}

to activate the virtual environment again to make \codename{mesh\_hydro\_utils}
available. You can automate that away by putting the command above in your
shell's runtime configuration file, which is usually \verb|$HOME/.bashrc| for
\jargon{bash}, \verb|$HOME/.zshrc| for \jargon{zsh} etc.









%======================================================
\subsection{Compiling This Documentation}
%======================================================


This documentation is stored within the repository in the
\verb|hydro_playground/tex/documentation| directory. You can build it using the
provided \verb|Makefile|, like so:


\begin{lstlisting}
$ cd hydro_playground/tex/documentation
$ make
\end{lstlisting}

That should leave you with the resulting \verb|documentation.pdf| file.

Alternately, you can run the latex compile command by hand:

\begin{lstlisting}
$ cd hydro_playground/tex/documentation
$ pdflatex -jobname=documentation documentation.tex
\end{lstlisting}

Another way yet is to import the documentation directory into your LaTeX editor
of choice. The main file you need to import, open, and compile is
\verb|documentation.tex|.









%======================================================
\subsection{Compiling the Code}
%======================================================

For the default setup, the code only requires \cmake (version 3.21 and
above) to build the project, and a C++ compiler compatible with the C++11
standard.

To build the code, navigate into the repository directory and run the usual
\cmake workflow (where we first create a directory \verb|build|, descend
into it, and then run the \cmake commands):

\begin{lstlisting}
$ cd hydro_playground
$ mkdir build
$ cd build
$ cmake ..
$ cmake --build .
\end{lstlisting}

An alternative, but equivalent workflow, is

\begin{lstlisting}
$ cd hydro_playground
$ cmake -B build
$ cmake --build build/
\end{lstlisting}


That should leave you with an executable file \verb|hydro| in the directory
\verb|hydro_playground/build/|.




%======================================================
\subsection{Python}
%======================================================

\todo{All of this}




%======================================================
\subsection{Running an Example}
%======================================================


\todo{All of this}


