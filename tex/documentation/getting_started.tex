\newpage
%======================================================
\section{Getting Started}
%======================================================



%======================================================
\subsection{Getting the Code}
%======================================================


The code is hosted on a github repository:
\url{https://github.com/mladenivkovic/hydro-playground}. You can get it by
cloning the repository using either

\begin{lstlisting}
$ git clone https://github.com/mladenivkovic/hydro-playground.git
\end{lstlisting}


using the https protocol, or

\begin{lstlisting}
$ git clone git@github.com:mladenivkovic/hydro-playground.git
\end{lstlisting}

using ssh.






%======================================================
\subsection{Compiling This Documentation}
%======================================================


This documentation is stored within the repository in the
\verb|hydro_playground/tex/documentation| directory. You can build it using the
provided \verb|Makefile|, like so:


\begin{lstlisting}
$ cd hydro_playground/tex/documentation
$ make
\end{lstlisting}

That should leave you with the resulting \verb|documentation.pdf| file.

Alternately, you can run the latex compile command by hand:

\begin{lstlisting}
$ cd hydro_playground/tex/documentation
$ pdflatex -jobname=documentation documentation.tex
\end{lstlisting}

Another way yet is to import the documentation directory into your LaTeX editor
of choice. The main file you need to import, open and compile is
\verb|documentation.tex|.









%======================================================
\subsection{Compiling the Code}
%======================================================

For the default setup, the code only requires \verb|cmake| (version 3.21 and
above) to build the project, and a C++ compiler compatible with the C++11
standard.

To build the code, navigate into the repository directory and run the usual
\verb|cmake| workflow (where we first create a directory \verb|build|, descend
into it, and then run the \verb|cmake| commands):

\begin{lstlisting}
$ cd hydro_playground
$ mkdir build
$ cd build
$ cmake ..
$ cmake --build .
\end{lstlisting}

That should leave you with an executable file \verb|hydro| in the directory
\verb|hydro_playground/build/|.



%======================================================
\subsection{Python}
%======================================================

\todo{All of this}




%======================================================
\subsection{Running an Example}
%======================================================


\todo{All of this}


