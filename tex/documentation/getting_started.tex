\newpage
%======================================================
\section{Getting Started}
%======================================================


%======================================================
\subsection{Requirements}
%======================================================

\begin{itemize}
\item   \code{git} to obtain the code.
\item   A good old \codename{C++} compiler. Code is written in \codename{C++17}
        standard.
\item   \cmake 3.21 or above
\item   (optional) \codename{Python 3} with \codename{numpy} and
        \codename{matplotlib} for plotting outputs and generating initial
        conditions.
\item   (optional) \codename{LaTeX} to create the \jargon{TeX} files. I
        hard-coded the \jargon{pdflatex} command in the scripts. It doesn't
        require any fancy \codename{LaTeX} packages.
\end{itemize}





%======================================================
\subsection{Getting the Code}
%======================================================


The code is hosted on a github repository:
\url{https://github.com/mladenivkovic/hydro-playground}. You can get it by
cloning the repository using either

\begin{lstlisting}
git clone https://github.com/mladenivkovic/hydro-playground.git
\end{lstlisting}


using the \jargon{https} protocol, or

\begin{lstlisting}
git clone git@github.com:mladenivkovic/hydro-playground.git
\end{lstlisting}

using \jargon{ssh}.








%======================================================
\subsection{Getting and Installing the Python Module}\label{chap:py}
%======================================================


The entire \mhutils python module is stored within this repository as a git
submodule of its own
repository\footnote{\url{https://github.com/mladenivkovic/mesh_hydro_utils}}.

Once you've cloned the \hydro repository, you'll also need to tell git to grab
the submodules using

\begin{lstlisting}
git submodule init
git submodule update
\end{lstlisting}

When completed successfully, the directory \verb|./python_module| should now
contain some files. We now need to install this python module.

The easiest way is to navigate into the directory and install it locally using
e.g. \code{pip}:

\begin{lstlisting}
cd python_module
pip install -e .
\end{lstlisting}

On your local machine, that should work just fine. If you are trying to install
this python package some place where your permissions are restricted, e.g. on a
cluster or HPC facility, then \code{pip} may be prohibited from installing this
way. In those cases, you can try

\begin{lstlisting}
pip install --user -e .
\end{lstlisting}

first. If that doesn't work, a reliable fail-safe is to create a python virtual
environment and install it there. (In fact, this should be the preferred
method.) You can do this by:

\begin{lstlisting}
python3 -m venv $HOME/virtualenv/some_name
source $HOME/virtualenv/some_name/bin/activate
cd ./python_module
pip install -e .
\end{lstlisting}

You may choose the path \verb|$HOME/virtualenv/some_name| freely to your liking.
Note that every time you start a new session, you'll need to run the command

\begin{lstlisting}
source $HOME/virtualenv/some_name/bin/activate
\end{lstlisting}

to activate the virtual environment again to make \codename{mesh\_hydro\_utils}
available. You can automate that away by putting the command above in your
shell's runtime configuration file, which is usually \verb|$HOME/.bashrc| for
\jargon{bash}, \verb|$HOME/.zshrc| for \jargon{zsh} etc.









%======================================================
\subsection{Compiling This Documentation}\label{chap:documentation}
%======================================================


This documentation is stored within the repository in the
\verb|hydro_playground/tex/documentation| directory. You can build it using the
provided \verb|Makefile|, like so:


\begin{lstlisting}
cd hydro_playground/tex/documentation
make
\end{lstlisting}

That should leave you with the resulting \verb|documentation.pdf| file.

Alternately, you can run the latex compile command by hand:

\begin{lstlisting}
cd hydro_playground/tex/documentation
pdflatex -jobname=documentation documentation.tex
\end{lstlisting}

Another way yet is to import the documentation directory into your LaTeX editor
of choice. The main file you need to import, open, and compile is
\verb|documentation.tex|.









%======================================================
\subsection{Compiling the Code}
%======================================================

For the default setup, the code only requires \cmake (version 3.21 and
above) to build the project, and a C++ compiler compatible with the C++11
standard.

To build the code, navigate into the repository directory and run the usual
\cmake workflow (where we first create a directory \verb|build|, descend
into it, and then run the \cmake commands):

\begin{lstlisting}
cd hydro_playground
mkdir build
cd build
cmake ..
cmake --build .
\end{lstlisting}

An alternative, but equivalent workflow, is

\begin{lstlisting}
cd hydro_playground
cmake -B build
cmake --build build/
\end{lstlisting}


That should leave you with an executable file \verb|hydro| in the directory
\verb|hydro_playground/build/|.







%======================================================
\subsubsection{Build Options}
%======================================================


You can pass build options to \cmake by giving it a list of command line
arguments beginning with \code{-D} at build time, e.g.

\begin{lstlisting}
cd hyrdo_playground
mkdir build
cd build
cmake .. -DOPTION1 -DOPTION2 ...
cmake --build .
\end{lstlisting}

or, if you prefer:

\begin{lstlisting}
cd hyrdo_playground
cmake -B build -DOPTION1 -DOPTION2 ...
cmake --build build
\end{lstlisting}


Currently available build options are:

\begin{itemize}
\item \verb|-DBUILD_TYPE=| [\code{Release}, \code{RelWithDebInfo}, \code{Debug}]
    \begin{itemize}
        \item  \code{Release}: Enables aggressive compiler optimisation. This
                is the default mode.
        \item  \code{RelWithDebInfo}: Release mode, but with debugging symbols
                attached. Also activates some light debugging checks.
        \item  \code{Debug}: Turns compiler optimisation off and enables
                extensive debugging checks.
    \end{itemize}
\end{itemize}


\todo{Precision flag, once it's implemented}







%======================================================
\subsection{Running an Example}
%======================================================


Once you've compiled the code following the steps in the previous sections,
you're ready to run your first example.

A successful compilation will leave you with an executable
\verb|hydro_playground/build/hydro|. To actually run the code, you need to
provide it with two mandatory command line arguments: A simulation parameter
file (Section~\ref{chap:paramfile}) and an initial conditions file
(Section~\ref{chap:icfile}).

You can specify them as follows:

\begin{lstlisting}
./hydro --ic-file <ic_file> --config-file <config_file>
\end{lstlisting}

or

\begin{lstlisting}
./hydro --ic-file=<ic_file> --config-file=<config_file>
\end{lstlisting}

where \verb|<ic_file>| is the path to the initial conditions file you want to
use, and \verb|<config_file>| is the path to the parameter file you want to use.

You may want to look into the \verb|hydro_playground/examples| directory for
some ready-to-go examples.



