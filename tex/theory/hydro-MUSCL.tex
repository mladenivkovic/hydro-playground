%=====================================================================
\subsection{The MUSCL-Hancock Method} \label{chap:MUSCL-hancock}
%=====================================================================



%======================================================
\subsubsection{Method}
%======================================================

The MUSCL-Hackock method is a method of the class of \textbf{M}onotone
\textbf{U}pwind \textbf{S}chemes for \textbf{C}onservation \textbf{L}aws, which
try to achieve higher order accuracy by describing the states $\U_i$ of each
cell $i$ not by a constant state, but by some higher order interpolation. The
MUSCL-Hanckock scheme in particular assumes a piecewise linear state
reconstruction (see Fig.~\ref{fig:piecewise-linear} for an example).


\begin{figure}[H]
\includegraphics[width=\textwidth]{./figures/piecewise_linear.pdf}%
\caption{
	A piecewise linear representation of continuous data among cells.
	\label{fig:piecewise-linear}
}
\end{figure}


It solves the hyperbolic conservation law of the form

\begin{equation*}
	\DELDT{\U} + \DELDX{\F(\U)} = 0
\end{equation*}

using the explicit conservative formula

\begin{equation}
\U_i^{n+1} =
	\U_i^n + \frac{\Delta t}{\Delta x} \left( \F_{i-\half} - \F_{i + \half}
\right)
\end{equation}

where $\U_i$ is the volume average of a state in a cell. However, $\U_i$ is
not assumed to be constant throughout the cell, but is described by

\begin{equation}
\U_{i}(x) =
	\U_i^n + \frac{x - x_i}{\Delta x} \mathbf{s}_i;
	\quad x \in [0, \Delta x]
\end{equation}


$\mathbf{s}_i$ is a suitably chosen slope vector of $\U_i(x)$ in cell $i$.
A general slope can be written as

\begin{equation}
\mathbf{s}_i =
 	\frac{1}{2} (1 + \omega) (\U_i - \U_{i-1}) +
 	\frac{1}{2} (1 - \omega) (\U_{i+1} - \U_i)
\end{equation}

with $\omega \in [-1, 1]$. For $\omega = 0$, we retrieve the centered scheme.
$\omega = 1$ gives us the upwind slope, $\omega = -1$ gives us the downwind
slope.


Having non-constant states creates a problem for Riemann solvers. We now need
to solve the so called \emph{Generalised Riemann Problem} with

\begin{align}
	\DELDT{\U} + \DELDX{\F(\U))} &= 0 \\
	\U(x, 0) &= 
	\begin{cases}
		\U_i(x), & \quad x < 0 \\
		\U_{i+1}(x), & \quad x > 0 \\
	\end{cases}
\end{align}

As the left and right states change with $x$, the characteristics are no longer
straight lines, which makes things tricky. So instead of dealing with that
analytically, the MUSCL-Hancock method tries to compute some sort of
``intermediate state'' such that in the end, we get an approximate flux that is
good enough for our purposes.

In each cell, the extreme values are located at the cell boundaries.
They are referred to as boundary extrapolated values, and are given by

\begin{equation}
	\U_i^L = \U_i^n - \frac{1}{2} \mathbf{s}_i;
	\quad
	\U_i^R = \U_i^n + \frac{1}{2} \mathbf{s}_i;
\end{equation}


To obtain a second order inter-cell flux, we first try and find intermediate
extrapolated boundary values by applying the same conservative explicit formula
on every cell separately over half the timestep. We denote the intermediate
boundary extrapolated values as $\overline{\U}_i^L$ and $\overline{\U}_i^R$,
and obtain them by computing

\begin{align}
\overline{\U}_i^L &=
\U_i^L +
\frac{1}{2} \frac{\Delta t}{\Delta x}
\left(
	\F(\U_i^L) - \F(\U_i^R)
\right)
\\
\overline{\U}_i^R &=
	\U_i^R +
	\frac{1}{2} \frac{\Delta t}{\Delta x}
	\left(
		\F(\U_i^L) - \F(\U_i^R)
	\right)
\end{align}

Note that this update depends only on the values inside a cell, and can be
computed for every cell individually.

Finally, with the evolved boundary extrapolated values, we can compute the
fluxes $\F_{i + \half} = \F(\U_{i+\half}(x = 0))$ by solving the Riemann
problem at every cell interface and using the initial values

\begin{align}
	\U_L &= \overline{\U}_i^R\\
	\U_R &= \overline{\U}_{i+1}^L 
\end{align}

and by sampling the solution at $ x = x_{i+\half}$.


Note that we only use the intermediate evolved state to estimate the evolved
boundary extrapolated values, which in their turn are used to obtain left and
right initial values for the Riemann problem. We don't use the evolved state
when finally evolving the state!

So in short, what we need to do is

\begin{itemize}

\item Compute slope $\mathbf{s}_i$ for each cell

\item find evolved boundary extrapolated values using
\begin{align*}
\overline{\U}_i^L &=
	\U_i^L +
	\frac{1}{2} \frac{\Delta t}{\Delta x}
	\left(
		\F(\U_i^L) - \F(\U_i^R) \right)
\\
\overline{\U}_i^R &=
	\U_i^R +
	\frac{1}{2} \frac{\Delta t}{\Delta x}
	\left(
		\F(\U_i^L) - \F(\U_i^R)
	\right)
\end{align*}

\item find fluxes $\F_{i + \half}$ as $\F(\U_{i+\half})$ with $\U_{i+\half}$
being the solution to the Riemann problem
\begin{align}
	\U_L &= \overline{\U}_i^R\\
	\U_R &= \overline{\U}_{i+1}^L
\end{align}
at $ x = x_{i+\half}$

\item update the state using
\begin{equation}
\U_i^{n+1} =
	\U_i^n +
	\frac{\Delta t}{\Delta x} \left( \F_{i-\half} - \F_{i + \half} \right)
\end{equation}
\end{itemize}




A TVD version is obtained by using slope limiters (Section~\ref{chap:limiters}),
i.e. replacing the slope $\mathbf{s}_i$ by a limited slope
$\overline{\mathbf{s}}_i$ with

\begin{align*}
\overline{\mathbf{s}}_i &=
	\xi(r) \mathbf{s}_i
\\
r &=
	\frac{\U_i - \U_{i-1}}{\U_{i+1} -
	\U_{i}}
	\quad\quad \text{for each component of }\U
\end{align*} 

some possible and implemented slope limiters $\xi(r)$ are given in
Section~\ref{chap:implemented_limiters}.
















%======================================================
\subsubsection{Algorithm Details}
%======================================================

Each cell contains 4 gas states:

\begin{enumerate}
\item The primitive state $\mathbf{W} = (\rho, \V, p)$ of the gas as \code{prim}
\item The flux of the primitive state $\F(\mathbf{W})$ as \code{pflux}
\item The conserved state $\U = (\rho, \rho \V, E)$ of the gas as \code{cons}
\item The flux of the conserved state $\F(\U)$ as \code{cflux}
\end{enumerate}


A flux at position $x_{i+\half, j}$ or $y_{i, j+\half}$ will be stored in the
cell with index $(i, j)$.


We can afford to store $x_{i+\half}$ at cell $i$ because we have at least 1
extra virtual boundary cell which is used to apply boundary conditions, so the
flux at $x_{-\half}$ will be stored in \verb|grid[BC-1]|, where \texttt{BC} is
the number of boundary cells used.


Additionally, we need to store the evolved extrapolated boundary states
$\overline{\U}_i^L$ and $\overline{\U}_i^R$ for each cell.


\todo{Adapt function name}

We treat the 2D solution using dimensional splitting
(Section~\ref{chap:dimensional-splitting}). Here, we describe the 1D algorithm.

The \verb|solver_step(...)| function does the following for the 1D case:

\begin{itemize}

\item Reset the stored fluxes from the previous timestep to zero

\item Compute the primitive states for all cells from the updated conserved
states

\item Impose boundary conditions (Section~\ref{chap:boundary-conditions})

\item Find the maximal timestep that you can do by applying the CFL condition
\ref{eq:godunov-cfl}.

\item Compute fluxes:

\begin{itemize}
\item For every cell, find the (limited) slope $\mathbf{s}_i$ and then compute
the evolved extrapolated boundary states $\overline{\U}_i^L$ and
$\overline{\U}_i^R$ for each cell.

We need to solve the Riemann problem with
$\U_L = \overline{\U}_i^R$, $\U_R = \overline{\U}_{i+1}^L$
later, so those values need to be present for the neighbouring cells before we
can compute the Riemann problem.

This is done by first computing the evolved boundary
extrapolated states in a single loop over all cells, and then the rest of the
solution is done in a second loop over all cells.

\item For every cell pair $(i, i+1)$, solve the Riemann problem (see
Section~\ref{chap:riemann}) to find the flux $\F_{i+\half}$ with $\U_L =
\overline{\U}_i^R$, $\U_R = \overline{\U}_{i+1}^L$.

\item Store the flux $\F_{i+\half}$ in cell $i$.

\end{itemize}

\item Update the states: Compute $\U^{n+1}$ using $\U_i^n$, the
flux $\F_{i+\half}$ stored in every cell $i$, and the flux $\F_{i-\half}$
stored in every cell $i-1$.

\end{itemize}
