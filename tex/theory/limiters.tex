\newpage
%============================================================
\section{Slope and Flux Limiters}\label{chap:limiters}
%============================================================



%===========================================
\subsection{Why Limiters?}
%===========================================

Limiters are employed because issues arise around numerical schemes due to
their discrete nature. For example, a non-limited piecewise linear advection
scheme will produce oscillations around jump discontinuities. See
\textbf{Godunov's Theorem}:


\begin{displayquote}
	Linear numerical schemes for solving partial differential equations
	(PDE's), having the property of not generating new extrema (monotone
	scheme), can be at most first-order accurate.
\end{displayquote}


So if we want to employ linear or higher order schemes such as the
MUSCL-Hancock method (Section~\ref{chap:MUSCL-hancock}), we will generate new
extrema, which induce non-physical extrema around steep gradients, in
particular around discontinuities.

A discussion how limiters do their magic is out of scope for this document. See
the references given in the Introduction for more details.





%=============================================================================
\subsection{ Implemented Limiters} \label{chap:implemented_limiters}
%=============================================================================

% =============================================================================
% \subsubsection{Flux limiters $\phi(r)$}
% %=============================================================================
%
% Some popular (and implemented) limiters are:
% \begin{flalign}
% \text{Minmod}
% 	&&\quad \phi(r) &=
% 	\mathrm{minmod}(1, r)
% \\
% \text{Superbee}
% 	&&\quad \phi(r) &= \max(0, \min(1, 2r), \min(2, r))
% \\
% \text{MC (monotonized cenral-difference)}
% 	&&\quad \phi(r) &= \max(0, \min ((1+r)/2, 2, 2r))
% \\
% \text{van Leer}
% 	&&\quad \phi(r) &= \frac{r + |r|}{1 + |r|}
% \end{flalign}
%
% where
%
% \begin{align}
% \mathrm{minmod}(a, b) =
% 	\begin{cases}
% 		a	& \quad \text{ if } |a| < |b| \text{ and } ab > 0\\
% 		b	& \quad \text{ if } |a| > |b| \text{ and } ab > 0\\
% 		0	& \quad \text{ if } ab \leq 0\\
% 	\end{cases}
% \end{align}
%
%





%=============================================================================
\subsubsection{Slope limiters $\xi(r)$}
%=============================================================================

\todo{Clear out what isn't implemented in \hydro}

Some popular (and implemented) limiters are:
\begin{flalign}
\text{Minmod}
	&&\quad \xi(r) &=
	\begin{cases}
		0, & r \leq 0\\
		r, & 0 \leq r \leq 1\\
		\min\{1, \xi_R(r)\}, & r \geq 1\\
	\end{cases}
\\
\text{Superbee}
	&&\quad \xi(r) &=
	\begin{cases}
		0, & r \leq 0\\
		2r, & 0 \leq r \leq \frac{1}{2}\\
		1, & \frac{1}{2} \leq r \leq 1 \\
		\min\{r, \xi_R(r), 2\}, & r \geq 1\\
	\end{cases}
\\
\text{van Leer}
	&&\quad \xi(r) &=
	\begin{cases}
		0, & r \leq 0\\
		\min\{\frac{2r}{1+r},  \xi_R(r)\}, & r \geq 0\\
	\end{cases}
\\
\end{flalign}

where

\begin{align}
\xi_R(r) &=
	\frac{2}{1 - \omega + (1 + \omega) r}
\\
r &=
	\frac{\U_i - \U_{i-1}}{\U_{i+1} - \U_{i}}
	\quad\quad\quad\quad \text{for each component of }\U
\end{align}



