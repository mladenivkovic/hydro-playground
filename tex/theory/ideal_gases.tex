\newpage
%===============================================
\section{Ideal Gases}\label{chap:ideal_gases}
%===============================================





%=======================================
\subsection{Governing Equations}
%=======================================

We are going to concern ourselves with ideal gases, which are described by the
Euler equations:


\begin{align}
	\deldt
	\begin{pmatrix}
		\rho \\
		\rho \V \\
		E
	\end{pmatrix}
	+
	\nabla \cdot
	\begin{pmatrix}
		\rho \V\\
		\rho \V \otimes \V + p\\
		(E + p) \V
	\end{pmatrix}
	=
	\begin{pmatrix}
		0\\
		\rho \mathbf{a}\\
		\rho \mathbf{a} \V
	\end{pmatrix}
\end{align}



Where
\begin{itemize}
\item $\rho$: fluid density
\item $\V$: fluid (mean/bulk) velocity at a given point. I use the notation
	$\V = (v_x, v_y, v_w)$, or when indices are useful, $\V = (v_1, v_2, v_3)$
\item $p$: pressure
\item $E$: specific energy. $E = \frac{1}{2} \rho \V^2 + \rho u$,
	where $u$ is the specific internal thermal energy
	(Eq.~\ref{eq:internal-energy})
\item $\mathbf{a}$: acceleration due to some external force.
\end{itemize}

The outer product $\cdot \otimes \cdot$ gives the following tensor:
\begin{align}
	(\V \otimes \V)_{ij} = v_i \cdot v_j
\end{align}



Furthermore, we have the following relations for ideal gasses:
\begin{align}
p &= n k T
	&& \text{Ideal Gas law} \\
p &= C \rho ^ \gamma
	&& \text{Entropy relation for smooth flow, i.e. no shocks} \\
s &= c_V \ln \left( \frac{p}{\rho^\gamma} \right) + s_0
	&& \text{Entropy}\\
a &= \sqrt{\frac{\del p}{\del \rho} \bigg{|}_s } = \sqrt{\frac{\gamma p}{\rho}}
	&& \text{Sound speed}
	\label{eq:soundspeed}
\end{align}

with 
\begin{itemize}
	\item $n$ : number density
	\item $k$ : Boltzmann constant
	\item $T$ : temperature
	\item $s$ : entropy
	\item $\gamma$: adiabatic index
	\item $c_V$: specific heat 
\end{itemize}

and the Equation of State

\begin{align}
	u &= \frac{1}{\gamma - 1}\frac{p}{\rho} \label{eq:internal-energy}
\end{align}




Neglecting any outer forces, i.e. $\mathbf{a} = 0$, the Euler equations can be
written as a conservation law as

\begin{align}
    \DELDT{ \U } + \nabla \cdot \F(\U) = 0
\end{align}

with

\begin{align}
	\U = 
		\begin{pmatrix}
			\rho \\ \rho \V \\ E
		\end{pmatrix}, &&
	\F(\U) = 
		\begin{pmatrix}
			\rho \V\\
			\rho \V \otimes \V + p\\
			(E + p) \V
		\end{pmatrix}
\end{align}








%===============================================
\subsubsection{Euler Equations in 1D}
%===============================================

In 1D, we can write the Euler equations without source terms ($\mathbf{a} = 0$)
as

\begin{align}
    \DELDT{ \U } + \DELDX{\F(\U)} = 0
\end{align}

or explicitly (using $\V = v$)


\begin{align}
	\deldt{ 
		\begin{pmatrix}
			\rho \\ \rho v \\ E
		\end{pmatrix}
		}
	+ \deldx {
		\begin{pmatrix}
			\rho v\\
			\rho v^2  + p\\
			(E + p) v
		\end{pmatrix}
	} = 0
\end{align}




%===============================================
\subsubsection{Euler equations in 2D}
%===============================================

In 2D, we have without source terms ($\mathbf{a} = 0$):



\begin{align}
\DELDT{ \U } + \DELDX{\F(\U)} + \frac{\del \mathbf{G}(\U)}{\del y} = 0
\label{eq:euler-dimensional-split}
\end{align}

or explicitly (using $\V = (v_x, v_y)$)

\begin{align}
	\deldt{ 
		\begin{pmatrix}
			\rho \\ \rho v_x \\ \rho v_y \\ E
		\end{pmatrix}
		}
	+ \deldx {
		\begin{pmatrix}
			\rho v\\
			\rho v_x^2  + p\\
			\rho v_x v_y\\
			(E + p) v_x
		\end{pmatrix}
	} 
	+ \frac{\del}{\del y}
		\begin{pmatrix}
			\rho v_y\\
			\rho v_x v_y\\
			\rho v_y^2  + p\\
			(E + p) v_y
		\end{pmatrix}
	= 0
\end{align}















%===============================================
\subsection{Conserved and Primitive Variables}
%===============================================


For now, we have described the Euler equation as a hyperbolic conservation law
using the (conserved) state vector $\U$:

\begin{align}
	\U = 
		\begin{pmatrix}
			\rho \\ \rho \V \\ E \ .
		\end{pmatrix}
\end{align}


For this reason, the variables $\rho$, $\rho \V$, and $E$ are referred to as
``conserved variables'', as they obey conservation laws.

However, this is not the only set of variables that allows us to describe the
fluid dynamics. In particular, the solution of the  Riemann problem (section
\ref{chap:riemann}) will give us a set of so called ``primitive
variables'' (or ``physical variables'') with the ``primitive'' state vector
$\mathbf{W}$:

\begin{align}
	\mathbf{W} = 
		\begin{pmatrix}
			\rho \\ \V \\ p
		\end{pmatrix}
\end{align}



For the ideal gas equations, these are the equations to translate between
primitive and conservative variables:

\textbf{Primitive to conservative:}

\begin{align}
	(\rho) &= (\rho) \\
	(\rho \V) &= (\rho) \cdot (\V) \\
	(E) &= \frac{1}{2} (\rho) (\V)^2 + \frac{(p)}{\gamma - 1} 
\end{align}


\textbf{Conservative to primitive:}

\begin{align}
	(\rho) &= (\rho) \\
	(\V) &= \frac{(\rho \V )}{(\rho)}\\
	(p) &= (\gamma - 1)  \left((E) - \frac{1}{2} \frac{(\rho \V)^2}{(\rho)} \right) 
\end{align}



