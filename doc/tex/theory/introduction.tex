% ====================================================
\section{Introduction}
% ====================================================


This document serves mainly to collect and explain the equations and methods
used to run the \hydro code. It is not intended to be an introduction to
physics, ideal gases, numerical hydrodynamics, or anything else really. For a
proper, extensive introduction, see e.g. \citet{toro} and \citet{leveque_2002}.

For a crash-course introduction into the topic, see e.g. \citet{gearrt}. Part I
of that document is an introduction into numerical hydrodynamics following
mostly \citet{toro}, with lots of the maths derived on a step-by-step basis. It
grew alongside the \codename{Mesh-Hydro}
code\footnote{\url{https://github.com/mladenivkovic/mesh-hydro}}, which is a
playground to learn about and experiment with numerical methods to solve
hyperbolic conservation laws (in particular the advection equation and the Euler
equations of ideal gases). The source files of \citet{gearrt} are also
publicly
available\footnote{\url{https://github.com/mladenivkovic/thesis_public}}.

For general implementation details, see the accompanying document in \\
\texttt{hydro-playground/tex/documentation}.



% ====================================================
\subsection{The Big Picture}
% ====================================================

To start with, here's some of the most important points to get you started:

\begin{itemize}
\item We're solving the Euler equations, i.e. the equations governing ideal
    gases (Section~\ref{chap:ideal_gases}).
\item We're solving them in 2D using Finite Volume methods.
\item This means that we discretise (split) a 2D domain into small (2D) cells,
    which we use to solve the equations.
\item All cells are of equal size. The entire domain is described as a uniform
    grid. We do not use anything remotely clever or sophisticated, such as
    adaptive refinement etc.
\item We solve the equations over small time intervals.
\item We call these time intervals ``steps'' or ``time steps''.
\item Solving the equations for all cells in a grid over a single time step is
    called a ``simulation step''.
\item The simulation (consecutive solution) of time steps evolves the state of
    the gas over time.
\item The simulation is ended once we reach our desired end time or number of
    steps.
\end{itemize}






% ==============================================
\subsection{Notation}
% ==============================================


We are working on numerical methods: Both space and time will be discretised.

Space will be discretised in cells which will have integer indices to describe
their position in the grid. Time will be discretised in fixed time steps, which
may have variable lengths. Nevertheless, the time progresses step by step.

The lower left corner has indices $(0, 0)$.
% In 1D, index 0 also represents the leftmost cell.




We have:
\begin{itemize}
    \item Integer subscript: Value of a quantity at the cell, i.e. the center
        of the cell. Example: $\U_i$, $\U_{i-2}$ or $\U_{i, j+1}$ for 2D.
    \item Non-integer subscript: Value at the cell faces, e.g. $\F_{i-\half}$ is
        the flux at the interface between cell $i$ and $i-1$, i.e. the left
        cell as seen from cell $i$.
    \item Integer superscript: Indication of the time step. E.g. $\U ^ n$:
        State at timestep $n$
    \item Non-integer superscript: (Estimated) value of a quantity in between
        timesteps. E.g. $\F^{n + \half}$: The flux at the middle of the time
        between steps $n$ and $n + 1$.
\end{itemize}
