\newpage
%======================================================
\section{General Implementation Details}
%======================================================



%-------------------------------------------------------
\subsection{Repository Layout}
%-------------------------------------------------------

\begin{itemize} 

    \item All the source files are in the \verb|src/| directory.

    \item \verb|tests/| contains unit tests and functional tests.
	
    \item \verb|doc/tex/| contains LaTeX documentation of the code:

    \begin{itemize}
        \item   \verb|doc/tex/documentation/| contains the code documentation,
                including how to get the code, first steps and getting started,
                and developer notes.

        \item   \verb|doc/tex/theory/| contains equations being solved, and the
                numerical methods used to solve them.

    \end{itemize}

    \item \verb|doc/rtd/| contains documentation regarding parallelisation
        efforts, instructions, and results.

    \item \verb|python_module/| is a git submodule containing the
                \mhutils python module.

    \item \verb|examples/| contains ready-to-go example simulations. See the
                README therein for more details.

\end{itemize}








%-------------------------------------------------------------------------------
\subsection{Contents of Specific Files and Directories in \texttt{src/}}
%-------------------------------------------------------------------------------

\todo{
    Do this once you've reached a stable version.
}








%-------------------------------------------------------
\subsection{How the Code Works}
%-------------------------------------------------------




\begin{itemize}

\item All the source files are in the \verb|src/| directory.

\item The main program file is \verb|src/main.cpp|

\item The program starts off by initialising some objects, reading the provided
command line arguments, and then reading in the parameter file and the initial
conditions (IC) file. Both these files are mandatory command line arguments,
i.e. the code will complain and shut down if you don't provide them.

\item While reading in the initial conditions, the grid (array of cells) is
allocated.

The grid contains an array of cells. Each cell holds a primitive and a
conserved gas state.

The grid is extended by a few cells on the outer boundaries to facilitate
boundary conditions. (That number is typically 2).

\item Global simulation parameters are held by an instance of the
\code{Parameters} class.

\item The main loop starts now:

\begin{itemize}
    \item Advance the solver for a time step. What exactly this entails depends
on the used solver.

    \item Write output data if you need to (when to dump output is specified in
the parameter file).

    \item Write a message to the screen after the time step is done.

    \item Update the current simulation time and continue until you reach the
end time.
\end{itemize}
\end{itemize}



To see what ``advance the solver for a time step'' means, refer to their
respective documentation in \code{tex/theory}.









% ============================================
\subsection{Additional Notes}
% ============================================


\begin{itemize}
\item The ``Grid'' contains a one-dimensional array of Cells
(\code{Cell.h: Cell()}). To access it correctly, use the provided getter
function
%
\begin{lstlisting}
in Cell.h: Cell().getCell(i, j);
\end{lstlisting}
%
Indices $(0, 0)$ represent the lower left corner of the simulation domain.


\item Each cell contains 4 gas states:

\begin{enumerate}
\item The primitive state $\mathbf{W} = (\rho, \V, p)$ of the gas as \code{prim}
\item The flux of the primitive state $\F(\mathbf{W})$ as \code{pflux}
\item The conserved state $\U = (\rho, \rho \V, E)$ of the gas as \code{cons}
\item The flux of the conserved state $\F(\U)$ as \code{cflux}
\end{enumerate}

A flux at position $x_{i+\half, j}$ or $y_{i, j+\half}$ will be stored in the
cell with index $(i, j)$.
We can afford to store $x_{i+\half}$ at cell $i$ because we have at least 1
extra virtual boundary cell which is used to apply boundary conditions, so the
flux at $x_{-\half}$ will be stored in \verb|[BC-1]|, where \texttt{BC} is
the number of boundary cells used.


\item All default values for parameters read from the parameter file are set in
the parameters constructor. If you want to add a new parameter, consult the
class documentation in \code{src/Parameters.h}.

\item To write log messages, consult the documentation in \code{src/Logging.h}.

\item To use timers internally, consult the documentation in \code{src/Timer.h}.

\end{itemize}
