\newpage
% ====================================================================
\section{File Format Specification}
% ====================================================================

% ====================================================================
\subsection{Parameter File Specification}\label{chap:paramfile}
% ====================================================================

\note{
    An example parameter file containing all recognized parameters is stored as
    \texttt{hydro\_playground/examples/example\_paramfile.txt}.
}


% ---------------------------------------
\subsubsection{Talking Parameters}
% ---------------------------------------

\begin{tabular}[c]{p{2.5cm} p{1.5cm} p{1.cm} p{9.cm}}
Name & Default & Type & Description \\
\hline
\hline
\texttt{verbose} &
    = 0 &
    \texttt{int} &
    How talkative the code should be. 0 = quiet, 1 = talky, 2 = no secrets, 3 = debugging
\\ \hline
\texttt{nstep\_log} &
    = 0 &
    \texttt{int} &
    Write log messages only ever \texttt{nstep\_log} steps. If 0, will write every step.
\\ \hline
\end{tabular}






% ---------------------------------------
\subsubsection{Simulation Parameters}
% ---------------------------------------

\begin{tabular}[c]{p{2.5cm} p{1.5cm} p{1.cm} p{9.cm}}
Name & Default & Type & Description \\
\hline
\hline
\texttt{nx} &
    = 0 &
    \texttt{int} &
    Number of cells to use if you're running with a two-state type IC file.
    Otherwise, it needs to be specified in the initial conditions. If you're not
    using a two-state IC, the value will be overwritten by the value given in
    the IC file.
\\ \hline
\texttt{ccfl} &
    = 0.9 &
    \texttt{float} &
    Courant factor; \texttt{dt\ =\ ccfl\ *\ dx\ /\ vmax}
\\ \hline
\texttt{nsteps} &
    = 0 &
    \texttt{int} &
    Up to how many steps to do. If = 0, run until \texttt{t}
    $\geq$ \texttt{tmax}
\\ \hline
\texttt{tmax} &
    = 0 &
    \texttt{float} &
    Up to which time to simulate. If \texttt{nsteps} is given, will stop running
    if \texttt{nsteps} steps are reached before \texttt{tmax} is.
\\ \hline
\texttt{boundary} &
    = 0 &
    \texttt{int} &
    Boundary conditions 0: periodic. 1: reflective. 2: transmissive. This sets
    the boundary conditions for all walls.
\\ \hline
\texttt{boxsize} &
    = 1. &
    \texttt{float} &
    Size of the simulation box in each dimension.
\\ \hline
\texttt{replicate} &
    = 0 &
    \texttt{int} &
    When running with arbitrary-type initial conditions, replicate (=
    copy-paste) the initial conditions this many times in each dimension. This
    is not used for the two-state type ICs because there you can simply specify
    the \texttt{nx} parameter as you wish.
\\ \hline
\end{tabular}







% ---------------------------------------
\subsubsection{Output Parameters}
% ---------------------------------------

\begin{tabular}[c]{p{2.5cm} p{1.5cm} p{1.cm} p{9.cm}}
Name & Default & Type & Description
\\
\hline
\hline
\texttt{foutput} &
    = 0 &
    \texttt{int} &
    Frequency of writing outputs in number of steps. If = 0, will only write
    initial and final steps.
\\ \hline
\texttt{dt\_out} &
    = 0 &
    \texttt{float} &
    Frequency of writing outputs in time intervals. Code will always write
    initial and final steps as well.
\\ \hline
% \texttt{toutfile} &
%     None &
%     \texttt{string} &
%     File name containing desired times (in code units) of output. Syntax of the
%     file: One float per line with increasing value.
% \\ \hline
\texttt{basename} &
    None &
    \texttt{string} &
    Basename for outputs. If not given, a basename will be generated based on
    compilation parameters and IC filename.
\\ \hline
\texttt{write\_replications} &
    = \texttt{false} &
    \texttt{bool} &
    If \texttt{replicate} $>$ 1, setting this to true will write the entire
    content of the box, including all replications.
\end{tabular}




















% ====================================================================
\subsection{Initial Conditions}\label{chap:icfile}
% ====================================================================



\hydro is able to read two types of IC files. In either case, they're expected
to be formatted text.

In both IC file types, lines starting with \texttt{//} or \texttt{/*} will be
recognized as comments and skipped. Empty lines are skipped as well.

Some example python scripts that generate initial conditions are given in
\verb|./python_module/scripts/IC|. Note that for the directory
\verb|./python_module/| to contain any files, you first need to initialize the
submodule. See the instructions Section~\ref{chap:py}.







% ====================================================================
\subsubsection{Two-State Type ICs}\label{chap:twostate-ic}
% ====================================================================

You can use a Riemann problem two-state initial condition file as follows:

\begin{lstlisting}
filetype = two-state
rho_L   = <float>
v_L     = <float>
p_L     = <float>
rho_R   = <float>
v_R     = <float>
p_R     = <float>
\end{lstlisting}

The line

\begin{lstlisting}
filetype = two-state
\end{lstlisting}

\textbf{must} be the first non-comment non-empty line, after which the
parameters \texttt{rho\_L}, \texttt{v\_L}, \texttt{p\_L}, \texttt{rho\_R},
\texttt{v\_R}, and \texttt{p\_R} follow.

The discontinuity between the changes will be in the middle along the x-axis.
Fluid velocity in y direction will be set to zero, \texttt{v\_L} and
\texttt{v\_R} will be set as \texttt{v\_x}.

Note: For ``historical'' reasons, the IC file reader also accepts \texttt{u\_L}
and \texttt{u\_r} for the velocities instead of \texttt{v\_L} and
\texttt{v\_r}, respectively.






% ====================================================================
\subsubsection{Arbitrary Type ICs}\label{chap:arbitrary-ic}
% ====================================================================

\note{
    The file format is not really suitable for display on a
    LaTeX-generated pdf. You may want to check out the
    README in the repository's root directory or on github.
}



You can provide an individual value for density, velocity, and pressure for each
cell. The IC file format is as follows:

The lines

\begin{lstlisting}
filetype = arbitrary
nx = <int>
ndim = <int>
\end{lstlisting}

\textbf{must} be the first non-comment non-empty lines, in that order.
\jargon{ndim} \textbf{must} be 2, as \hydro only runs
in 2D.

The IC format is as follows:

\begin{lstlisting}
filetype = arbitrary
nx = <integer, number of cells in any dimension>
ndim = 2
<density in cell (0, 0)>       <x velocity in cell (0, 0)>      <y velocity in cell (0, 0)>        <pressure in cell (0, 0)>
<density in cell (1, 0)>       <x velocity in cell (1, 0)>      <y velocity in cell (1, 0)>        <pressure in cell (1, 0)>
                                     .
                                     .
                                     .
<density in cell (nx-1, 0)>     <x velocity cell (nx-1, 0)>     <y velocity in cell (nx-1, 0)>     <pressure in cell (nx-1, 0)>
<density in cell (0, 1)>        <x velocity in cell (0, 1)>     <y velocity in cell (0, 1)>        <pressure in cell (0, 1)>
<density in cell (1, 1)>        <x velocity in cell (1, 1)>     <y velocity in cell (1, 1)>        <pressure in cell (1, 1)>
                                     .
                                     .
                                     .
<density in cell (nx-1, 1)>     <x velocity cell (nx-1, 1)>     <y velocity in cell (nx-1, 1)>     <pressure in cell (nx-1, nx-1)>
                                     .
                                     .
                                     .
<density in cell (0, nx-1)>     <x velocity in cell (0, nx-1)>  <y velocity in cell (0, nx-1)>     <pressure in cell (0, nx-1)>
<density in cell (1, nx-1)>     <x velocity in cell (1, nx-1)>  <y velocity in cell (1, nx-1)>     <pressure in cell (1, nx-1)>
                                     .
                                     .
                                     .
<density in cell (nx-1, nx-1)>  <x velocity cell (nx-1, nx-1)>  <y velocity in cell (nx-1, nx-1)>  <pressure in cell (nx-1, nx-1)>
\end{lstlisting}

\texttt{cell\ (0,\ 0)} is the lower left corner of the box. First index is x
direction, second is y. All values for density, velocity, and pressure must be
floats. You can put comments and empty lines wherever you feel like it.











% ====================================================================
\subsection{Output}
% ====================================================================

\note{
    The file format is not really suitable for display on a
    \code{LaTeX}-generated \jargon{pdf}. You may want to check out the
    \jargon{README} in the repository's root directory or on github.
}



If no \texttt{basename} is given in the parameter file, the output file
name will be generated as follows:

\begin{lstlisting}
output_XXXX.out
\end{lstlisting}

where \texttt{XXXX} is the snapshot/output number.


The output files are written in plain text, and their content should be
self-explanatory:

\begin{lstlisting}
# ndim =  2
# nx =    <number of cells used>
# t =     <current time, float>
# nsteps =  <current step of the simulation>
#            x            y          rho          v_x          v_y            p
<x value of cell (0, 0)> <y value of cell (0, 0)> <density in cell (0, 0)> <x velocity in cell (0, 0)> <y velocity in cell (0, 0)> <pressure in cell (0, 0)>
<x value of cell (1, 0)> <y value of cell (1, 0)> <density in cell (1, 0)> <x velocity in cell (1, 0)> <y velocity in cell (1, 0)> <pressure in cell (1, 0)>
                                                 .
                                                 .
                                                 .
<x value of cell (nx-1, 0)> <y value of cell (nx-1, 0)> <density in cell (nx-1, 0)> <x velocity cell (nx-1, 0)> <y velocity in cell (nx-1, 0)> <pressure in cell (nx-1, 0)>
<x value of cell (0, 1)> <y value of cell (0, 1)> <density in cell (0, 1)> <x velocity in cell (0, 1)> <y velocity in cell (0, 1)> <pressure in cell (0, 1)>
<x value of cell (1, 1)> <y value of cell (1, 1)> <density in cell (1, 1)> <x velocity in cell (1, 1)> <y velocity in cell (1, 1)> <pressure in cell (1, 1)>
                                                 .
                                                 .
                                                 .
<x value of cell (nx-1, 1)> <y value of cell (nx-1, 1)> <density in cell (nx-1, 1)> <x velocity cell (nx-1, 1)> <y velocity in cell (nx-1, 1)> <pressure in cell (nx-1, nx-1)>
                                                 .
                                                 .
                                                 .
<x value of cell (0, nx-1)> <y value of cell (0, nx-1)> <density in cell (0, nx-1)> <x velocity in cell (0, nx-1)> <y velocity in cell (0, nx-1)> <pressure in cell (0, nx-1)>
<x value of cell (1, nx-1)> <y value of cell (1, nx-1)> <density in cell (1, nx-1)> <x velocity in cell (1, nx-1)> <y velocity in cell (1, nx-1)> <pressure in cell (1, nx-1)>
                                                 .
                                                 .
                                                 .
<x value of cell (nx-1, nx-1)> <y value of cell (nx-1, nx-1)> <density in cell (nx-1, nx-1)> <x velocity cell (nx-1, nx-1)> <y velocity in cell (nx-1, nx-1)> <pressure in cell (nx-1, nx-1)>
\end{lstlisting}


