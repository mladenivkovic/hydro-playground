\newpage


%======================================================
\section{Introduction}
%======================================================


\hydro is a simple toy code solving finite volume hydrodynamics in 2D on a
uniform grid. The main goal is to have simple working code as a playground for
developments in terms of acceleration, optimisation, and parallelisation.

It is based on the \codename{Mesh-Hydro}
code\footnote{\url{https://github.com/mladenivkovic/mesh-hydro}}, which is a
playground to learn about and experiment with numerical methods to solve
hyperbolic conservation laws, in particular the advection equation and the Euler
equations of ideal gases.

\hydro is written in C++ and makes use of \codename{cmake} for the build system.
A simple python module, \mhutils\footnote{
\url{https://github.com/mladenivkovic/mesh\_hydro\_utils}}, provides convenience
functions to plot outputs and generate initial conditions. It depends on
\codename{matplotlib} and \codename{numpy}.

This documentation only handles the base, unparallelised solver. For the
individual parallelisation strategies, notes, and results, consult the
documentation in \verb|hyrdo-playground/doc/rtd|.


